\chapter{Հրամանային տողի արգումենտներ}

\begin{quote}
Այս զրույցը Սի լեզվով գրված ծրագրերին հրամանային տողից
փոխանցված արգումենտների մասին է։
\end{quote}

Գործարկման պահին հրամանային տողում տրված արգումենտները ծրագրում
հասանելի դարձնելու համար պետք է \ident{main} ֆունկցիան (ծրագրի
մուտքի կետը) սահմանել \code{int} և \code{char**} տիպի պարամետրերով։

\begin{Verbatim}
int main( int, char** );
\end{Verbatim}

Այդ պարամետրերից առաջինը, որին ավանդաբար ընդունված է տալ
\ident{argc} (argument count -- արգումենտների քանակ) անունը,
ցույց է տալիս հրամանային տողի բաղադրիչների քանակը՝ ներառյալ
ծրագիր անունը։ Օրինակ, եթե \texttt{prog09a.c} ծրագիրը
կոմպիլյացնելու համար հրամանային տողում գրում եմ.

\begin{Verbatim}
clang prog09a.c -o prog09a
\end{Verbatim}

ապա \texttt{clang} ծրագրի \ident{main} ֆունկցիայի առաջին
(\ident{argc}) պարամետրը կստանա 4 արժեքը։

\ident{main} ֆունկցիայի երկրորդ պարամետրը, որ հայտարարվել է
\code{char**} տիպով, հրամանային տողի բաղադրիչների հասցեների զանգվածն
է։ Քանի որ արգումենտները հրամանային տողից փոխանցվում են տեղերի
տեսքով, իսկ տողի տիպը \code{char*} է, ապա այդ տողերի հասցեների
զանգվածի տիպը կլինի \code{char**}։ Այս երկրորդ պարամետրին էլ
ավանդաբար ընդունված է տալ \ident{argv} (argument values --
արգումենտների արժեքներ) անունը։

Հրամանային տողից տրված արգումենտների օգտագործումը ցուցադրելու
համար գրեմ մի ծրագիր, որը հերթականությամբ արտածում է հրամանային
տողում տրված արգումենտները և դրանց ինդեքսները։

\begin{Verbatim}
#include <stdio.h>

int main( int argc, char** argv )
{
  for( int c = 0; c < argc; ++c )
    printf( "%d: %s\n", c, argv[c] );

  return 0;
}
\end{Verbatim}

Այս ծրագիրը պահպանում եմ \texttt{prog09a.c} ֆայլում, կոմպիլյացնում
եմ սովորականի պես և գործարկում եմ ահա այսպես․

\begin{Verbatim}
$ ./prog08b abcd 123 'efgh 45'
\end{Verbatim}

Աշխատանքի արդյունքում ստանում եմ հետևյալ արտածումը, որից երևում
է, որ \ident{argc} պարամետրը ստացել է 4 արժեքը, իսկ \ident{argv}
ցանգվածի տարրերը ցույց են տալիս համապատասխանաբար \texttt{./prog09a},
\texttt{abcd}, \texttt{123} և \texttt{efgh 45} տողերին։

\begin{Verbatim}
0: ./prog09a
1: abcd
2: 123
3: efgh 45
\end{Verbatim}

Այսքանը, թերևս, բավական է C լեզվով գրված ծրագրերում հրամանային
տողից տրված արգումենտների հետ աշխատանքի մասին պատկերացում կազմելու
համար։ Բայց ես ուզում եմ մի օրինակ էլ բերել, որտեղ բացի այն,
որ ծրագիրն օգտագործում է հրամանային տողի արգումենտները, այլև
դրանք օգտագործում է իր ճիշտ վարքը կազմակերպելու համար։

Ես գրում եմ մի ծրագիր, որը հրամանային տողի երկրորդ արգումենտով
տրված տասական թիվը ձևափոխում է երկուական, ութական կամ տասնվեցական
ներկայացման։ Ձևափոխման ֆորմատը տրվում է հրամանային տողի առաջին
արգումենտով։

Ծրագրում օգտագործելու եմ ֆունկցիաներ, որոնք հայտարարված են
\texttt{ctype.h}, \texttt{math.h}, \texttt{stdio.h} և
\texttt{string.h} ֆայլերում։

\begin{Verbatim}
#include <ctype.h>
#include <math.h>
#include <stdio.h>
#include <string.h>
\end{Verbatim}

Առաջին գործը պետք է լինի ստուգել, որ հրամանային տողում տրված
են ճիշտ երկու արգումենտներ։ Քանի որ \ident{argc} պարամետրը
հաշվում է նաև ծրագիր անունը, ապա այն պետք է համեմատել 3-ի հետ։

\begin{Verbatim}
int main(int argc, char** argv)
{
  /* Հրամանային տողում պետք է տալ անպայման երկու արգումենտ */
  if( argc != 3 ) {
    puts("Սպասվում է ճիշտ երկու արգումենտ։");
    printf("%s <ֆորմատ> <թիվ>\n", argv[0]);
    return 1;
  }
\end{Verbatim}

Առաջին արգումենտի առաջին նիշը ցույց է տալիս նոր ներկայացման
ֆորմատը։ Այն կարող է լինել «\texttt{b}», «\texttt{o}» կամ
«\texttt{h}»։ Ստուգում եմ նաև այդ պայմանը։

\begin{Verbatim}
  /* Թվի ֆորմատը ցույց տվող պարամետրը */
  char format = argv[1][0];

  /* Առաջին արգումենտը պետք է լինի «b» «o» կամ «h» */
  if( 'b' != format && 'o' != format && 'h' != format ) {
    puts("Սխալ ֆորմատի արգումենտ։ Պետք է լինի․");
    puts("  b - երկուական,");
    puts("  օ - ութական,");
    puts("  հ - տասնվեցական։");
    return 2;
  }
\end{Verbatim}

Ձևափոխվող թիվը, որ տրված է երկրորդ արգումենտով, պետք է
բաղկացած լինի միայն տասական թվանշաններից։

\begin{Verbatim}
  /* Երկրորդ արգումենտը պետք է բաղկացած լինի միայն թվանշաններից */
  for( int i = 0; i < strlen(argv[2]); ++i )
    if( !isdigit(argv[2][i]) ) {
      puts("Թիվը պետք է բաղկացած լինի միայն տասական թվանշաններից։");
      return 3;
    }
\end{Verbatim}

Բոլոր ստուգումներն արդեն արված են։ Եթե ծրագիրը կհայտնաբերի
սխալներ կան ոչ ճիշտ մուտքային տվյաներ, ապա դրանց մասին կտրվի
համապատասխան հայտարարություն և ծրագրի աշխատանքը կավարտվի՝
օպերացիոն համակարգին վերադարձնելով համապատասխան սխալի կոդ։

Հրամանային տողից ձևափոխվող թիվը ծրագրին հասանելի է տողի
տեսքով։ Որպեսզի նրա հետ հնարավոր լինի աշխատել որպես
\emph{թիվ}, \ident{scanf} ֆունկցիայով \Verb|argv[2]| տողից
ստանում եմ \code{int} արժեք։

\begin{Verbatim}
  /* Ձևափոխվելիք տասական թիվը */
  int dec_number = 0;
  sscanf(argv[2], "%d", &dec_number);
\end{Verbatim}

Հիմա բուն ձևափոխությունները։ Սի լեզվի \ident{printf} ֆունկցիան
կարողանում է տասական թիվը ձևափոխել ութական և տասնվեցական տեսքերի.
պետք է պարզապես օգտագործել համապատասխանաբար \Verb|%o| և \Verb|%x|
ֆորմատավորման հրահանգները։

\begin{Verbatim}
  if( 'o' == format )
    printf("0%o\n", dec_number);
  else if( 'h' == format )
    printf("0x%x\n", dec_number);
\end{Verbatim}

Բայց թվի երկուական տեսքը ստանալու համար պետք է մի քիչ աշխատել։
Հայտնի է, որ թվի տասական ներկայացումից երկուական ներկայացումը
ստանալու համար պետք է հակառակ կարգով վերցնել նրա՝ երկուսին
բաժանելուց ստացված մնացորդները։ Եվ որպեսզի այդ մնացորդները
գրեմ ճիշտ հաջորդականությամբ՝ աջից ձախ, պետք է իմանամ, թե թվի
երկուական ներկայացման համար քանի դիրք է հարկավոր։ Այդ նիշերի
քանակը ստանալու համար պետք վերցնել տասական թվի երկու հիմքով
լոգարիթմը։ Դրանից հետո պարզապես պետք է \ident{dec\_number}
թիվը բաժանել երկուսի այնքան անգամ, քանի դեռ այն չի հավասարվել
զրոյի։

\begin{Verbatim}
  else if( 'b' == format ) {
    /* Թվի երկուական նիշերի քանակը */
    int nc = log(dec_number)/log(2.0);
    /* Ձևափոխված թվի բուֆեր */
    char str_number[32] = { 0 };
    /* Ձևափոխություն երկուական տեսքի */
    while( dec_number != 0 ) {
      str_number[nc] = '0' + dec_number % 2;
      dec_number /= 2;
      --nc;
    }
    /* Արդյունքի արտածում */
    printf("%s\n", str_number);
  }

  return 0;
}
\end{Verbatim}

\textbf{Վարժություն:} Գրել թեստերի խումբ, որն ամբողջությամբ
ստուգում է այս վերջին ծրագիրը։
