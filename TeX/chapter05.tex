\chapter{Ստրուկտուրաներ}

\begin{quote}
\emph{?}
\end{quote}

Մինչ այժմ, երբ օրինակներում փոփոխականներ կամ ֆունկցիաներ էի հայտարարում,
օգտագործում էի Սի լեզվի պարզ (սկալյար) տիպերը՝ \kw{int}, \kw{double},
\kw{float} և այլն։ Բայց, առավել հաճախ կարիք է լինում ծրագրերում օգտագործել
բաղադրյալ օբյեկտներ, այնպիսիք, որոնց կառուցվածքը սահմանում է ծրագրավորողը։
Օրինակ, հենց նույն այդ դեկարտյան կետերի հետ աշխատելիս ավելի հարմար կլինի
ունենալ մի օբյեկտ, որն ունի \ident{x} և \ident{y} հատկությունները
(դաշտերը), քան աշխատել երկու \kw{double} թվերի կամ նրանց ցուցիչների հետ։

Սի լեզվի \kw{struct} ծառայողական բառը ծրագրավորողին տալիս է այդ
հնարավորությունը։ Օրինակ.

\begin{Verbatim}
struct { double x, y; } p0;
\end{Verbatim}

Այս հրամանով \ident{p0} փոփոխականը հայտարարեցի որպես \ident{x} և \ident{y}
դաշտերն ունեցող բաղադրյալ օբյեկտ՝ \emph{ստրուկտուրա}։

Ստրուկտուրայի դաշտերին դիմում են \Verb|.| գործողության միջոցով։ Օրինակ,
\ident{p0} օբյեկտը տպելու համար պետք է գրել.

\begin{Verbatim}
printf( "%lf, %lf", p0.x, p0.y );
\end{Verbatim}

Ստրուկտուրային օբյեկտներ սահմանելիս \kw{struct} ծառայողական բառից հետո
կարելի է գրել ստրուկտուրայի անուն, որն էլ հետագայում կարելի է օգտագործել
նույն տիպի այլ օբյեկտներ սահմանելիս։ Օրինակ, \ident{p1} դեկարտյան կետի
սահմանումը կարող է ունենալ այսպիսի սահմանում.

\begin{Verbatim}
struct point { double x, y; } p1;
\end{Verbatim}

Այս տիպի արտահայտությունը թույլ է տալիս ծրագում ամեն անգամ
\texttt{struct \{ double x, y; \}} գրելու փոխարեն պարզապես գրել
\texttt{struct point}։

Հիմա հետ վարադառնամ իմ հին խնդիրն, երբ գրեցի \ident{polar} ֆունկցիան, որը
տրված կետի դեկարտյան կոորդինատներից հաշվում է դրա բևեռային կոորդինատները։
Նախ հայտարարեմ \ident{cartesian\_point} ստրուկտուրան՝ իր \ident{x},
\ident{y} դաշտերով, և \ident{polar\_point} ստրուկտուրան՝ \ident{rho},
\ident{phi} դաշտերով։

\begin{Verbatim}
/* դեկարտյան կետը */
struct cartesian_point {
  double x; /* աբսցիս */
  double y; /* օրդինատ */
};

/* բևեռային կետը */
struct polar_point {
  double rho; /* շառավիղ */
  double phi; /* ազիմուտ */
};
\end{Verbatim}

ՈՒնենալով \ident{cartesian\_point} և \ident{polar\_point} ստրուկտուրաների
հայտարարությունները, ես կարող եմ սահմանել \ident{to\_polar} ֆունկցիան, որը
ստանում է դեկարտյան կետ և վերադարձնում է համարժեք բևեռային կետը։

\begin{Verbatim}
struct polar_point to_polar( struct cartesian_point c )
{
  struct polar_point res;
  res.rho = sqrt( c.x * c.x + c.y * c.y );
  res.phi = atan2( c.y, c.x );
  return res;
}
\end{Verbatim}

Ճիշտ նույն կերպ կարող եմ սահմանել \ident{to\_cartesian} ֆունկցիան, որը
ստանում է բևեռային կետ և վերադարձնում է դրա դեկարտյան ներկայացումը։

\begin{Verbatim}
struct cartesian_point to_cartesian( struct polar_point p )
{
  struct cartesian_point res;
  res.x = p.rho * cos( p.phi );
  res.y = p.rho * sin( p.phi );
  return res;
}
\end{Verbatim}

Հիմա գրեմ \ident{main} ֆունկցիան, որտեղ ցուցադրված են վերը հայտարարված
ստրուկտուրաների ու սահմանված ֆունկցիաների կիրառությունները։

\begin{Verbatim}
int main()
{
  struct cartesian_point c0;
  puts( "Ներածիր դեկարտյան կոորդինատները x,y․ " );
  scanf( "(%lf, %lf)", &c0.x, &c0.y );

  struct polar_point p0 = to_polar( c0 );
  puts( "Բևեռային կոորդնատներն են․ " );
  printf( "ρ = %lf, φ = %lf\n", p0.rho, p0.phi );

  struct cartesian_point c1 = to_cartesian( p0 );
  puts( "Դեկարտյան կոորդնատներն են․ " );
  printf( "x = %lf, y = %lf\n", c1.x, c1.y );

  return 0;
}
\end{Verbatim}

Ակնհայտ է, որ \ident{cartesian\_point} և \ident{polar\_point} ստրուկտուրաները,
ըստ էության, նույնն են։ Երկուսն էլ մոդելավորում են կետի գաղափարը, բայց
տարբերվում են իրենց դաշտերի անուններով։ Կարելի է երկուսի փոխարեն սահմանել
մեկ \ident{point} ստրուկտուրա՝ դաշտերին տալով \ident{first} (առաջին) և
\ident{second} (երկրորդ) պայմանական անունները, և այդ միակ ստրուկտուրան
օգտագործել և՛ դեկարտյան, և՛ բևեռային կետերը ներկայացնելու համար։ Օրինակ.

\begin{Verbatim}
struct point {
  double first;
  double second;
};
\end{Verbatim}

Դեկարտյան կետերը բևեռային կետերից տարբերելու համար կարող եմ \kw{typedef}
հրամանով սահմանել \texttt{struct point} տիպի երկու \emph{հոմանիշ անուններ}։

\begin{Verbatim}
typedef struct point cartesian;
typedef struct point polar;
\end{Verbatim}

\kw{typedef} հրամանը ծրագրում սահմանում է արդեն գոյություն ունեցող տիպի
համարժեք, հոմանիշ անուն։ Օրինակ, կարող եմ իմ ծրագրում \ident{real} անունը
սահմանել որպես \kw{long} \kw{double} տիպի անուն․

\begin{Verbatim}
typedef long double real;
\end{Verbatim}

Առավել հաճախ \kw{typedef} հայտարարությունն օգտագործվում է հենց ստրուկտուրաների
համար կարճ անուններ հայտարարելու նպատակով։ Մեկ \kw{typedef} արտահայտությամբ
կարելի է մի քանի համարժեք անուններ։ Օրինակ.

\begin{Verbatim}
typedef struct point {
  double first;
  double second;
} cartesian, polar;
\end{Verbatim}

Ավելին, \kw{typedef} հայտարարություններ անելիս, կարելի է բաց թողնել
ստրուկտուրայի անունը (եթե այն այլևս չի օգտագործվելու)։ Այսպես.

\begin{Verbatim}
typedef struct {
  double first;
  double second;
} cartesian, polar;
\end{Verbatim}

Նման հայտարարությունից հետո վերը սահմանված \ident{to\_polar} ֆունկցիան
կունենա հետևյալ տեսքը.

\begin{Verbatim}
polar to_polar( cartesian c )
{
  polar res;
  res.first = sqrt( c.first * c.first + c.second * c.second );
  res.second = atan2( c.second, c.first );
  return res;
}
\end{Verbatim}

Ստրուկտուրայի նոր նմուշ ստեղծելիս դրա դաշտերին տրվող արժեքները կարելի
է թվարկել \Verb|{| և \Verb|}| փակագծերի մեջ առնված ցուցակում։ Օրինակ,
\ident{to\_polar} ֆունկցիան կարելի է սահմանել նաև հետևյալ կերպ.

\begin{Verbatim}
polar to_polar( cartesian c )
{
  polar res = {
    sqrt( c.first * c.first + c.second * c.second ),
    atan2( c.second, c.first )
  };
  return res;
}
\end{Verbatim}

Եթե ինչ-որ պատճառով պետք է խախտել արժեքների թվարկման հաջորդականությունը,
ապա պետք է նշել, թե հերթական արժեքը որ դաշտին է վերագրվում՝ անունից առաջ
\Verb|.| նիշը գրելով։ Օրինակ, այսպես.

\begin{Verbatim}
struct cartesian c1 = { .y = 1.2, .x = 3.4 };
\end{Verbatim}

Բայց մի անհարմար բան էլ կա. ստիպված եմ և՛ \ident{cartesian}, և՛
\ident{polar} օբյեկտների դաշտերին դիմելու համար օգտագործել նրանց
\ident{first} և \ident{second} անունները, որի պատճառով գրված կոդը նվազ
հասկանալի է դառնում։ Ես այդ խնդիրը կարող եմ լուծել մի քանի պարզ
\emph{մակրոսներ} սահմանելով.

\begin{Verbatim}
#define X(p) p.first
#define Y(p) p.second
#define RHO(p) p.first
#define PHI(p) p.second
\end{Verbatim}

Կոմպիլյացիայից առաջ Սի լեզվով գրված ծրագրերը մշակվում են նախապրոցեսորի
կողմից, և մակրոսները հենց նախապրոցեսորի սահմանումներ են։ Մակրոսի
սահմանումն ունի հետևյալ տեսքը․

\[
\mathbf{\#define} \langle name\rangle[(\langle parameters\rangle)] \langle body\rangle
\]

Ամեն անգամ, երբ նախապրոցեսորը ծրագրի տեքստում հանդիպում է
\(\langle name\rangle\) անունը, դա փոխարինվում է մակրոսի
\(\langle body\rangle\) մարմնով՝ իհարկե, համապատասխան տեղերում տեղադրելով
մակրոսի պարամետրերի արժեքները։ C լեզվում ընդունված է մակրոսների անունները
գրել մեծատառերով։

Դեկարտյան և բևեռային կոորդինատների դաշտերը ընտրող մակրոսները սահմանելուց
հետո, օրինակ, \ident{to\_cartesian} ֆունկցիան կարող եմ սահմանել հետևյալ կերպ.

\begin{Verbatim}
cartesian to_cartesian( polar p )
{
  cartesian res;
  X(res) = RHO(p) * cos( PHI(p) );
  Y(res) = RHO(p) * sin( PHI(p) );
  return res;
}
\end{Verbatim}

Երբ C լեզվի նախապրոցեսորը կմշակի այս տեքստը, բոլոր մակրոսները կփոխարինվեն
իրենց սահմանումներով և կոմպիլյացիան կշարունակվի այնպես, կարծես թե մակրոսներ
չեն էլ եղել։

Ստրուկտուրայի չափը՝ հիշողություն մեջ նրա նմուշի զբաղեցրած բայթերի քանակը,
որոշվում է որպես նրա դաշտերի չափերի գումար։ Քանի որ \ident{point}
ստրուկտուրան ունի երկու \kw{double} տիպի դաշտեր, ապա ճիշտ է
\texttt{sizeof(struct point) >= 2 * sizeof(double)} հավասարությունը։
