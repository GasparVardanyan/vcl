\chapter*{Բան ներածական}

Հարգելի՛ ընթերցող։

Այս գրքում ես ուզում եմ խոսել Սի (C, /si:/) ծրագրավորման լեզվի մասին անպես, ինչպես կուզենայի, որ ինձ հետ խոսեին այն ժամանակ, երբ ես դեռ նոր էի մասնագիտական կրթություն ստանում։ Այս գիրքը նախատեսել եմ հատկապես սկսնակ ծրագրավորողների համար և նպատակ ունեմ, հնարավորությանս սահմաններում, լրացնել ծրագրավորման տեխնիկային վերաբերող հայալեզու գրականության պակասը։ Բնականաբար, իմ նպատակը ծրագրավորման Սի լեզուն ամբողջությամբ նկարագրելը չէ, և այս գրքույկը չի կարող Սի լեզվի տեղեկատուի դեր կատարել։ Ես պատմում եմ ամենաանհրաժեշտի մասին, երբեմն նաև այնպիսի բաների մասին, որոնք քիչ են հանդիպում, կամ չեն հանդիպում դասագրքերում։

Ծրագրավորման Սի լեզվի մասին գրելու պատճառներից ու նախադրյալներից մեկը դրա պահանջված ու արդիկան լինելն է։ Կարելի է, օրինակ, հետևել ծրագրավորման լեզուների տարածվածության [TIOBE Index](http://www.tiobe.com/index.php/content/paperinfo/tpci/index.html) վարկանիշին, որն ամեն ամիս ինչ-որ չափանիշերով որոշում և հրապարակում է այն 20 (ապա 50) ծրագրավորման լեզուների աղյուսակը, որոնց մասին ամենաշատն են հարցումներ կատարվել որոնողական համակարգերում։ Վերջին 15 տարիներին այդ վարկանիշում Սի լեզուն զբաղեցնում է կա՛մ առաջին, կա՛մ երկրորդ հորիզոնականը։ Չնայած վարկանշային այս եղանակը բավականին քննադատվում է, բայց, այնուամենայնիվ, որոշակի պատկերացում տալիս է արդի ծրագրավորման լեզուների կիրառության մասին։

Նախաձեռնությանս պատճառներից մյուսը Սի լեզվի քերականությունն է, որն, անկեղծ ասած, ինձ այնքան էլ դուր չի գալիս։ Բայց այսօր լայն տարածում գտած լեզուներից շատերը, ինչպիսիք են, օրինակ, Java, Go և այլ, որպես հիմք են ընդունել Սի լեզվի քերականությունը։ Եվ չնայած այն բանին, որ ամեն մի ծրագրավորման լեզու ունի իր ինքնուրույն գաղափարները, Սի լեզվի ուսումնասիրությունն ինչ-որ չափով կհեշտացնի նման քերականությամբ այլ լեզուներ սովորելը։ ???

Սին կոմպակտ, լավ ուսումնասիրված, հարուստ գրադարաններով ծրագրավորման լեզու է, նրա մասին գրվել և գրվում են բազմաթիվ հաջող գրքեր (ցավոք, դրանց մեջ չկան հայերենով գրված կամ հայերեն թարգմանված օրինակներ)։ ???

Գրքի բոլոր գլուխները պարունակում են օրինակներ։ Ավելին, գիրքը կառուցված է օրինակների վրա, և այդ բոլոր օրինակներն ամբողջական, աշխատող ծրագրեր են։ Ես փորձում եմ անել այնպես, որ ընթերցողի մոտ հարցեր չմնան, թե ինչպես աշխատեց, կամ ինչի համար է գրված ծրագրի այս կամ այն կտորը։ Հընթացս ես պատմում եմ նաև օգտագործված օժանդակ գրադարաննների, գործիքների կամ մեթոդների մասին։

Գիրքն ապրում է [github.com/armenbadal/vcl](https://github.com/armenbadal/vcl) էջում։ Այստեղ են նաև գրքի տեքստում որպես օրինակներ ներկայացված ծրագրերը։
