\chapter{Տեքստի տողեր}

\begin{quote}
Այս զրույցը տողերի մասին է։ Նախ՝ համառոտ կպատմեմ տողերի
ներկայացման և դրանց հետ առավել հաճախ կատարվող գործողությունների
մասին, ապա մի ամբողջական օրինակով կցուցադրեմ այդ նկարագրված
հնարավորությունների օգտագործումը։
\end{quote}

Սի լեզվում \emph{տողը} նիշերի (character) միաչափ զանգված է, որի
վերջին տարրն անպայման պետք է \texttt{0} արժեքը (կամ \verb|\0|
նիշը) լինի։ Տողի՝ \Verb|\0| նիշով ավարտվելու հատկությունն ակտիվորեն
օգտագործվում է տողերի հետ աշխատող ֆունկցիաներում։ Օրինակ, տողի
երկարությունը հաշվելու համար բավական է հաշվել տողի սկզբից մինչև
առաջին \Verb|\0| նիշն ընկած նիշերի քանակը։ Ասենք, եթե
\texttt{string.h} գրադարանում արդեն սահմանված չլիներ \ident{strlen}
 ֆունկցիան, ապա սկսնակ ու անփորձ ծրագրավորողը կարող էր այն
 իրականացնել մոտավորապես հետևյալ կերպ.

\begin{Verbatim}
int strlen0( char* str )
{
  int count = 0;               /* նիշերի քանակի հաշվիչ */
  while( str[count] != '\0' )  /* քանի դեռ նիշը \0 չէ */
    ++count;                   /* ավելացնել հաշվիչը */
  return count;
}
\end{Verbatim}

Ավելի փորձառու ծրագրավորողը նույնպես կօգտագործի \Verb|\0| նիշի
առկայությունը, բայց կօգտագործի ցուցիչների հետ կատարվող թվաբանություն։

\begin{Verbatim}
size_t strlen1( const char* s )
{
  const char* p = s;      /* սկսել ցուցակի սկզբից */
  while( *++p != '\0' );  /* քանի դեռ p-ն \0 նիշի վրա չէ, առաջ տանել այն */
  return p - s;           /* վերադարձնել երկու ցուցիչների տարբերությունը */
}
\end{Verbatim}

Օգտագործված \kw{size\_t} տիպը նախատեսված է օբյեկտների չափը
ցույց տվող փոփոխականներ սահմանելու համար։ Նրա արժեքը առանց
նշանի ամբողջ թիվ է։

Մեկ ուրիշն էլ տողի՝ \Verb|'\0'| նիշով ավարտվելու հատկությունը
կօգտագործի որպես ռեկուրսիայի ավարտի հայտանիշ ու կգրի հետևյալը.

\begin{Verbatim}
size_t strlen2( const char* s )
{
  if( '\0' == *s ) return 0;
  return 1 + strlen2(s + 1);
}
\end{Verbatim}

Հիմա տեսնենք, թե ինչպես են սահմանվում ու արժեքավորվում տողերը։
Դե, քանի որ տողը նիշերի զանգված է, ապա «\texttt{String}» տողը
պարունակող \ident{h0} փոփոխականի սահմանումը կարող է ունենալ
հետևյալ տեսքը.

\begin{Verbatim}
char h0[] = { 'S', 't', 'r', 'i', 'n', 'g', '\0' };
\end{Verbatim}

Սի լեզվում տողային հաստատունը ներկայանում է \Verb|"| չակերտների
մեջ վերցրած նիշերի հաջորդականությամբ և կոմպիլյատորը թույլ է տալիս
նիշերի զանգվածն արժեքավորել հանց այդ տիպի հաստատուններով։ Այսինքն,
վերը բերված սահմանմանն է համարժեք հետևյալը.

\begin{Verbatim}
char h1[] = "String";
\end{Verbatim}

Այս դեպքում տողն ավարտող \Verb|'\0'| նիշը բացահայտ տրված չէ,
բայց կոմպիլյատորը միշտ ավելացնում է այն։

Այնուհետև, քանի որ զանգվածի անունը ցուցիչ է իր առաջին տարրին,
կարող եմ գրել նաև այսպիսի սահմանում.

\begin{Verbatim}
char* h3 = "String";
\end{Verbatim}

Երբ տողի սահմանման ժամանակ չափը բացահայտ տրվում է, ապա պետք է
տեղ նախատեսել նաև տողի վերջի \Verb|'\0'| նիշի համար։ Օրինակ,
եթե «\texttt{String}» տողի երկարությունը 6 է, ապա պետք է գրել.

\begin{Verbatim}
char h2[7] = "String";
\end{Verbatim}

Եթե \ident{s0} և \ident{s1} փոփոխականները հայտարարված են որպես
տողեր (նույնն է թե նիշերի զանգվածներ կամ նիշերի ցուցիչներ), ապա
պարզ է, որ \ident{s0}-ի արժեքը \ident{s1}-ին վերագրելու համար
\ident{s1 = s0} արտահայտությունը բավարար չէ։ Այս դեպքում \ident{s1}
ցուցիչին կվերագրվի \ident{s0} ցուցիչի արժեքը և այդ երկուսն էլ
ցույց կտան հիշողության միևնույն տիրույթին։

Տողերի վերագրման կամ, ավելի ճիշտ ասած՝ \emph{պատճենման}, համար
նախատեսված են գրադարանային \ident{strcpy} և \ident{strncpy}
ֆունկցիաները։ \ident{strcpy} ֆունկցիան իր երկրորդ արգումենտի
պարունակությունը պատճենում է առաջինի մեջ (պատճենելով նաև տողն
ավարտող \Verb|'\0'| նիշը)։ Օրինակ.

\begin{Verbatim}
char* h4 = "One string";
char h5[11] = { 0 };
strcpy( h5, h4 );
puts( h5 ); /* արտածվում է One string */
\end{Verbatim}

Տողեր պատճենող \ident{strncpy} ֆունկցիան իր աճաջին արգումենտի
մեջ է պատճենում երկրորդ արգումենտում տրված տողի նշված քանակով
նիշեր։ Պատճենվող նիշերի քանակը տրվում է ֆունկցիայի երրորդ
արգումենտով։ Օրինակ, այսպես.

\begin{Verbatim}
char* h6 = "0123456789";
char h7[11] = { 0 };
strncpy( h7, h6, 5 );
puts( h7 ); /* արտածվում է 01234 */
\end{Verbatim}

Եթե պետք է մի տողին \emph{կցել} մեկ այլ տողի պարունակություն, ապա
օգտագործվում են \ident{strcat} և \ident{strncat} ֆունկցիաները։
Առաջինը կցում է ամբողջ տողը, իսկ երկրորդը՝ միայն տրված քանակով
նիշեր։ Օրինակ.

\begin{Verbatim}
char s4[64] = { 0 };
strcpy( s4, "One" );
puts( s4 ); /* արտածվում է One */
strcat( s4, ", Two" );
puts( s4 ); /* արտածվում է One, Two */
strncat( s4, ", Three, Four", 7 );
puts( s4 ); /* արտածվում է One, Two, Three */
\end{Verbatim}

Տողերը \emph{համեմատվում} են \ident{strcmp} ֆունկցիայով։ Այս
ֆունցկիան կատարում է արգումենտում տրված երկու տողերի
\emph{բառարանային} համեմատում։ Եթե երկու տողերի պարունակությունը
համընկնում է \ident{strcmp} ֆունկցիան վերադարձնում է \(0\)։ Եթե
առաջինի արժեքը բառարանային կարգով ավելի փոքր է երկրորդի արժեքից,
ապա վերադարձվում է բացասական արժեք, իսկ եթե մեծ է՝ դրական արժեք։
\ident{strncmp} ֆունկցիայի երրորդ արգումենտով նշվում է, թե երկու
տողերի սկզբից քանի նիշ պետք է համեմատել։

\ident{strchr} ֆունկցիան իր առաջին արգումենտում տրված տողում
որոնում է երկրորդ արգումենտում տրված նիշը։ Այս ֆունկցիան
վերդարձնում է գտնված նիշի ցուցիչը, կամ \ident{NULL}՝ եթե
այդպիսին չի գտնվել։ Օրինակ.

\begin{Verbatim}
char* s0 = "One String";
char* s1 = strchr( s0, 'S' );
puts( s1 ); /* արտածում է String */
\end{Verbatim}

Տողի մեջ մի այլ տող որոնելու համար է նախատեսված \ident{strstr}
ֆունկցիան։ Այն իր առաջին արգումենտում տրված տողի մեջ որոնում է
երկրորդ արգումետում տրված ենթատողը։ \ident{strstr} ֆունկցիան
վերադարձնում է գտնված ենթատողի առաջին նիշի ցուցիչը, կամ
\ident{NULL}՝ եթե ենթատողը չի գտնվել։ Օրինակ.

\begin{Verbatim}
char* s2 = "One long string";
char* s3 = strstr( s2, "long" );
puts( s3 ); /* արտածում է long string */
\end{Verbatim}

* * *

Տողերի հետ աշխատող ֆունկցիաները ցուցադրելու համար ներկայացնեմ
բավականին հայտնի մի խնդիր։ Տրված է \(1..999999\) միջակայքի մի
դրական ամբողջ թիվ և պահանջվում է կառուցել նրա բառային
արտահայտությունը։ Օրինակ, եթե տրված է \(428\), ապա պետք է
կառուցել «չորս հարյուր քսանութ» տողը։ Այս խնդրի լուծման ալգորիթմն
այնքան պարզ է, որ նույնիսկ ամոթ էլ է դրան «ալգորիթմ» ասել։

Եվ այսպես, թող որ \ident{num} թվի թվային տեսքից բառային տեսքը
ստացող ֆունկցիան ունի հետևյալ հայտարարությունը, առաջին արգումենտը
թիվն է, իսկ երկրորդ արգումենտն այն բուֆերն է, որտեղ պետք է
գրվի պատասխանը։

\begin{Verbatim}
void number_as_words( int, char* );
\end{Verbatim}

Մի քիչ առաջ ընկնելով սահմանեմ \ident{number\_as\_words} ֆունկցիան
տեստավորող \ident{test\_number\_as\_words} ֆունկցիան, որը ստանում
է թիվ և այդ թվի սպասվող բառային տեսքը։

\begin{Verbatim}
void test_number_as_words( int number, const char* expected )
{
  char result[128] = { 0 }; /* պատասխանի բուֆեր */
  number_as_words( number, result ); /* թարգմանություն */
  if( 0 != strcmp( result, expected ) )  /* համեմատում սպասվող արժեքի հետ */
    printf( "ՍԽԱԼ։ %d թվի համար\n սպասվում է '%s'\n  ստացվել է '%s'\n",
        number, expected, result );
}
\end{Verbatim}

Ունենալով \ident{number\_as\_words} և \ident{test\_number\_as\_words}
ֆունկցիաները, իմ ծրագրում պետք է կատարվեն հետևյալ արտահայտությունները
և ոչ մի սխալի հաղորդագրություն չպետք է արտածվի։

\begin{Verbatim}
test_number_as_words( 1, "մեկ" );
test_number_as_words( 9, "ինն" );
test_number_as_words( 11, "տասնմեկ" );
test_number_as_words( 96, "ինսունվեց" );
test_number_as_words( 964, "ինն հարյուր վաթսունչորս" );
test_number_as_words( 9610, "ինն հազար վեց հարյուր տաս" );
test_number_as_words( 24531, "քսանչորս հազար հինգ հարյուր երեսունմեկ" );
test_number_as_words( 110310, "մեկ հարյուր տաս հազար երեք հարյուր տաս" );
\end{Verbatim}

Հիմա հենց \ident{number\_as\_words} ֆունկցիայի մասին։ Նախ սահմանեմ
երկու գլոբալ ստատիկ զանգվածներ, որոնք օգտագործվելու են ֆունկցիայի
մեջ։ Դրանցից առաջինը պարունակում է միանիշ թվերի անունները (բացի
զրոյից, որովհետև նրա անունը պետք չի գալիս), իսկ երկրորդը պարունակում
է տասնավորների անունները։

\begin{Verbatim}
static const char* ones[] = {
  "", "մեկ", "երկու", "երեք", "չորս",
  "հինգ", "վեց", "յոթ", "ութ", "ինն" };

static const char* tens[] = {
  "", "տաս", "քսան", "երեսուն", "քառասուն", "հիսուն",
  "վաթսուն", "յոթանասուն", "ութսուն", "ինսուն" };
\end{Verbatim}

Նաև սահմանեմ \ident{ord} մակրոսը, որը պետք է օգտագործեմ թվանշանի
նիշային (char) տեսքից նրա տասական արժեքը ստանալու համար։ ASCII
աղյուսակում թվանշաները դասավորված են հերթականորեն՝ \texttt{0},
\texttt{1},..., \texttt{9}։ Իսկ Սի լեզվում նիշերի հետ թվաբանական
գործողություններ կատարելիս օգտագործվում են դրանց ASCII կոդերը։
Հետևաբար, \ident{c} թվանշանի թվային արժեքը ստանալու համար բավական
է նրա կոդից հանել \texttt{0} նիշի կոդը։

\begin{Verbatim}
#define ord(c) (c-'0')
\end{Verbatim}

Քանի որ ձևափոխությունը սահմանափակվելու է միայն \([1..999999]\)
միջակայքի թվերի համար, \ident{sprintf} ֆունկցիայով ֆորմատավորում
եմ տրված թիվը՝ հատկացնելով նրան 6 դիրք։

\begin{Verbatim}
void number_as_words( int num, char* wnum )
{
  char snum[7] = { 0 };
  sprintf( snum, "%6d", num );
\end{Verbatim}

Այս հրամաննի կատարումից հետո, օրինակ, \(304\) թիվը \ident{snum}
բուֆերում կունենա \Verb|␣␣␣304\0| տեսքը։ (Այստեղ \Verb|␣| նիշն
օգտագործել եմ \emph{բացատ} նիշի դիրքերը ցույց տալու համար։)

![թիվը](number.svg)

Հերթականորեն վերլուծում եմ \ident{snum} բուֆերի \(0\)-ից \(5\)
ինդեքսով դիրքերը և \ident{wnum} տողին եմ կցում համապատսխան
արտահայտությունը։

Եթե \Verb|snum[0] != ' '|, ապա պետք է արտածել միավորի անունը և
«հարյուր» բառը։ Միավորի անունը վերցնում եմ \ident{ones} վեկտորից՝
արպես ինդեքս օգտագործելով \Verb|snum[0]| թվանշանի թվային արժեքը։

\begin{Verbatim}
  if( snum[0] != ' ' ) {
    strcpy( wnum, ones[ord(snum[0])] );
    strcat( wnum, " հարյուր " );
  }
\end{Verbatim}

Եթե \Verb|snum[1] != ' '|, ապա պետք է արտածել համապատասխան տասնյակի
անունը՝ միայն «տաս» բառից հետո ավելացնելով «ն» տառը։

\begin{Verbatim}
  if( snum[1] != ' ' ) {
    strcat( wnum, tens[ord(snum[1])] );
    if( snum[1] == '1' && snum[2] != '0' )
        strcat( wnum, "ն" );
  }
\end{Verbatim}

Եթե \Verb|snum[2] != ' '|, ապա պետք է արտածել միավորի անունը։

\begin{Verbatim}
  if( snum[2] != ' ' )
    strcat( wnum, ones[ord(snum[2])] );
\end{Verbatim}

Այս կետում ավարտեցի թվի առաջին կեսի վերլուծությունը։ Հիմա պետք է
ստուգել արդյո՞ք վերլուծվող թիվը մեծ է հազարից։ Եթե այդպես է, ապա
արտածել «հազար» բառը` երկու կողմերից ավելացնելով բացատանիշեր։

\begin{Verbatim}
  if( num >= 1000 )
    strcat( wnum, " հազար " );
\end{Verbatim}

Շարունակում եմ թվանշանների վերլուծությունը թվի երկրորդ կեսի համար,
որը ճիշտ նման է առաջին կեսի վերլուծությանը՝ այն տարբերությամբ միայն,
որ \ident{snum} վեկտորի \(0\), \(1\) և \(2\)` ինդեքսների փոխարեն
օգտագործվում են համապատասխանաբար \(3\), \(4\) և \(5\)` ինդեքսները։

\begin{Verbatim}
  if( snum[3] != ' ' ) {
    strcat( wnum, ones[ord(snum[3])] );
    if( snum[3] != '0' || num < 1000 )
      strcat( wnum, " հարյուր " );
  }

  if( snum[4] != ' ' ) {
    strcat( wnum, tens[ord(snum[4])] );
    if( snum[4] == '1' && snum[5] != '0' )
        strcat( wnum, "ն" );
  }

  if( snum[5] != ' ' )
    strcat( wnum, ones[ord(snum[5])] );
}
\end{Verbatim}

Այսքանը։ Հիմա \ident{number\_as\_words} ֆունկցիան թվի թվանշանային
տեքսից ստանում է նրա բառային արտահայտությունը։ Բայց ակնհայտ է, որ
այս ֆունկցիան ունի լավացնելու տեղեր։ Մասնավորապես, կարելի է թվի
առաջին ու երկրորդ կեսը թարգմանող նույնական կոդը ձևակերպել որպես
առանձին ֆունկցիա։ Թող դա էլ մնա որպես վարժություն ընթերցողին։
