\chapter{Ֆայլեր}

%\begin{quote}
%?
%\end{quote}

Նախորդ զրույցներում ներկայացված օրինակներում ծրագիրն իր
տվյալները կարդում էր ներմուծման ստանդարտ հոսքից և իր
աշխատանքի արդյունքներն ուղարկում էր արտածման ստանդարտ
հոսքին՝ այդ նպատակների համար օգտագործելով \ident{stdio}
գրադարանի, օրինակ, \ident{scanf}, \ident{puts} և
\ident{printf} ֆունկցիաները։ Բայց «իսկական» ծրագրերում
տվյալները հաճախ պահպանվում են \emph{ֆայլերում} (կամ տվյալների
բազաներում), և «իսկական» ծրագրավորողը պետք է տիրապետի
ծրագրավորման լեզուների այն միջոցներին, որոնք հնարավորություն
են տալիս \emph{կարդալ} գոյություն ունեցող ֆայլի պարունակությունը
և \emph{ստեղծել} նոր ֆայլեր՝ դրանց մեջ \emph{գրելով} պահպանման
ենթակա տվյալները։

Սի ծրագրավորման լեզվում ֆայլի հետ աշխատանքը կազմված է երեք
հիմնական քայլերից․ \textbf{ա)} \emph{բացել} ֆայլը և ստանալ նրա
\emph{դեսկրիպտորը}, բոլոր այլ գործողությունները կատարվելու են
ֆայլային դեսկրիպտորի հետ \textbf{բ)} ստացված գեսկրիպտորն
օգտագործելով որպես ֆայլի հետ աշխատելու «կապ»՝ գրել ֆայլի մեջ
կամ կարդալ նրանից, \textbf{գ)} \emph{փակել} ֆայլի դեսկրիպտորը՝
ազատելով դրա զբաղեցրած ռեսուրսները։

Ստանդարտ գրադարանի `stdio.h` վերնագրային ֆայլում սահմանված
է \ident{FILE} տիպը։ Ֆայլերի հետ աշխատող ֆունկցիաներն օգտագործում
են \ident{FILE} տիպի ցուցիչը որպես ֆայլերի դեսկրիպտոր։ Օրինակ,
եթե ուզում եմ \ident{inp} անունը հայտարարել որպես ֆայլի դեսկրիպտոր,
պետք է գրեմ․

\begin{Verbatim}
FILE* inp = NULL;
\end{Verbatim}

Տրված անունով ֆայլի դեսկրիպտորը ստանալու համար պետք է օգտագործել
ֆայլը բացելու \ident{fopen} ֆունկցիան։ Այս ֆունկցիայի առաջին
արգումենտը բացվելիք ֆայլի անունն է, իսկ երկրորդը՝ ֆայլի հետ
աշխատելու \emph{ռեժիմը}։ Ահա նրա հայտարարությունը․

\begin{Verbatim}
FILE* fopen(const char*, const char*);
\end{Verbatim}

Եթե ֆայլը բացելը ձախողվել է, ապա \ident{fopen} ֆունկցիան վերադարձնում
է \ident{NULL} արժեքը։

Ֆայլը կարելի է բացել \emph{կարդալու}, \emph{գրելու} կամ
\emph{լրացնելու} ռեժիմներում։ Կարդալու ռեժիմում բացելու համար
\ident{fopen} ֆունկցիայի երկրորդ արգումենտը պետք է տալ \Verb|"r"|
(read բառից)։ Օրինակ, ուզում եմ \texttt{example0.txt} անունով
ֆայլը բացել կարդալու համար․

\begin{Verbatim}
const char* name = "example0.txt";
FILE* inp = fopen(name, "r");
if( inp == NULL )
    printf("ՍԽԱԼ։ ՝%s՝ ֆայլի բացելը ձախողվեց։\n", name);
\end{Verbatim}

Գրելու և լրացնելու ռեժիմներով ֆայլը բացելու համար պետք է
\ident{openf} ֆունկցիայի երկրորդ արգումենտում տալ համապատասխանաբար
\Verb|"w"| (write) և \Verb|"a"| (append)։

Ֆայլի դեսկրիպտորը փակելու համար ստանդարտ գրադարանը տրամադրում է
\ident{fclose} ֆունկցիան։ Ահա հայտարարությունը․

\begin{Verbatim}
int fclose(FILE*);
\end{Verbatim}

Հիմա այն մասին, թե ինչպես կարդալ ֆայլային հոսքից և գրել դրա մեջ։
Եթե հոսքը բացված է կարդալու համար, ապա \ident{fgetc} ֆունկցիան
հնարավորություն է տալիս դրանից կարդալ մեկ նիշ։ Այս ֆունկցիան
արգումենտում ստանում է հոսքի դեսկրիպտորը և վերադարձնում է դրանից
կարդացած հերթական նիշը։

\begin{Verbatim}
int fgetc(FILE*);
\end{Verbatim}

Եթե ֆայլում էլ կարդալու բան չկա՝ կարդալու գործողությունը հասել
է ֆայլի վերջին, ապա \ident{fgetc} ֆունկցիան վերադարձնում է
\ident{EOF} (end of file --- ֆայլի վերջը) հաստատուն արժեքը։

Օրինակ, \ident{inp} փոփոխականին կապված հոսքից բոլոր նիշերը
կարդալու է արտածման հոսքին ուղարկելու համար կարող եմ գրել
հետևյալ ցիկլը։

\begin{Verbatim}
char c = '\0';
while( EOF != (c = fgetc(inp)) )
    putchar(c);
\end{Verbatim}

Ֆայլի վերջը ստուգելու համար ստանդարտ գրադարանն ունի
\ident{feof} ֆունկցիան։ Դրա օգտագործմամբ վեր բերված ցիկլը
կարող եմ գրել հետևյալ կերպ։

\begin{Verbatim}
char c = fgetc(inp);
while( !feof(inp) ) {
    putchar(c);
    c = fgetc(inp);
}
\end{Verbatim}

Գրելու կամ լրացնելու համար բացված ֆայլային հոսքում մեկ նիշ
գրելու համար է նախատեսված ստանդարտ գրադարանի \ident{fputc}
ֆունկցիան։ Սրա առաջին արգումենտը նիշն է, իսկ երկրորդը՝
դեսկրիպտորը։ Եթե գրելու գործողությունը հաջողվում է, ապա
\ident{fputc} ֆունկցիան վերադարձնում է գրած նիշը, հակառակ
դեպքում՝ \ident{EOF} արժեքը։

Եթե, օրինակ, `inp` դեսկրիպտորը կապված է կարդալու հոսքի հետ,
իսկ \ident{out} դեսկրիպտորը՝ գրելու, ապա \ident{inp}-ի
պարունակությունը \ident{out}-ի մեջ պատճենելու համար պարզապես
պետք է գրել․

\begin{Verbatim}
char c = fgetc(inp);
while( !feof(inp) ) {
    fputc(c, out);
    c = fgetc(inp);
}
\end{Verbatim}

Այսպիսով, ես արդեն բավականաչափ նյութ պատմեցի, որպեսզի ցույց
տամ, թե ինչպես կարելի է գրել Linux համակարգերում ֆայլը պատճենող
\texttt{cp} հրամանի «նմանակ» ծրագիրը։ Ստորև ներկայացնում եմ
ամբողջական ծրագիրը՝ համապատասխան մեկնաբանություններով։

\begin{Verbatim}
/* ներմուծման/արտածման ստանդարտ գրադարան */
#include <stdio.h>

/**/
int main(int argc, char** argv)
{
  /* ծրագիրը սպասում է ճիշտ երկու արգումենտ */
  if( argc != 3 ) {
    printf("Սխալ արգումենտների քանակ։\nՊետք է գրել․\n");
    printf("  %s <ֆայլ> <պատճեն>\n", argv[0]);
    return 1;
  }

  /* պատճենվող ֆայլը */
  FILE* inp = fopen(argv[1], "r");
  if( inp == NULL ) {
    printf("ՍԽԱԼ։ `%s` ֆայլի բացելը ձախողվեց։\n", argv[1]);
    return 2;
  }

  /* պատճենի ֆայլը */
  FILE* out = fopen(argv[2], "w");
  if( out == NULL ) {
    printf("ՍԽԱԼ։ `%s` ֆայլի ստեղծումը ձախողվեց։\n", argv[2]);
    fclose(inp); /* փակել արդեն բացված դեսկրիպտորը */
    return 3;
  }

  /* պատճենելու ցիկլը */
  char c = fgetc(inp);  /* կարդալ առաջին նիշը*/
  /* քանի դեռ ֆայլի վերջը չէ */
  while( !feof(inp) ) {
    fputc(c, out); /* գրել նիշը */
    c = fgetc(inp); /* կարդալ հերթական նիշը */
  }

  /* փակել դեսկրիպտորները */
  fclose(out);
  fclose(inp);

  return 0;
}
\end{Verbatim}

Ֆայլի պարունակության հետ նիշ առ նիշ աշխատելու համար, իհարկե,
բավարար են \ident{fgetc} և \ident{fputc} ֆունկցիաները։ Դրանց
հաջորդական կանչերով նույնիսկ կարելի է ֆայլից ամբողջական տողեր
կարդալ, կամ ֆայլում ամբողջական տողեր գրել։ Բայց ստանդարտ
գրադարանում սահմանված \ident{fgets} և \ident{fputs} ֆունկցիաները
հենց նախատեսված են այդ գործը հեշտացնելու համար։

\ident{fgets} ֆունկցիան \ident{stream} հոսքից կարդում է
առավելագույնը \texttt{count-1} քնակությամբ նիշեր ու դրանք
գրում \ident{buffer} զանգվածում։ Հաջողության դեպքում
վերադարձնում է \ident{buffer}-ը, իսկ ձախողման դեպքում՝
\ident{NULL}։

\begin{Verbatim}
char* fgets(char* buffer, int count, FILE* stream);
\end{Verbatim}

ident{fputs} ֆունկցիան շատ նման է արդեն հիշատակված \ident{puts}
ֆունկցիային, որը տրված տողը դուրս էր բերում արտածման ստանդարտ
հոսքին։ \ident{fputs} ֆունկցիան \ident{str} տողն արտածում է տրված
\ident{stream} հոսքին։ Ձախողման դեպքում այս ֆունկցիան վերադարձնում
է \ident{EOF} արժեքը, իսկ հաջողության դեպքում՝ ոչ-բացասական թիվ։

\begin{Verbatim}
int fputs(const char* str, FILE* stream);
\end{Verbatim}

Ֆորմատավորված ներմուծման ու արտածման համար նույնպես նախատեսված
են \ident{scanf} և \ident{printf} ֆունկցիաների ֆայլային
տարբերակները՝ \ident{fscanf} և \ident{fprintf} ֆունկցիաները։

\begin{Verbatim}
int fscanf(FILE* stream, ...);
int fprintf(FILE* stream, const char* format, ...);
\end{Verbatim}

Ի դեպ, ավելի ճիշտ է ասել, որ հենց \ident{scanf}, \ident{printf}
և \ident{puts} ֆունկցիաներն են \ident{fscanf}, \ident{fprintf}
և \ident{fputs} ֆունկցիաների մասնավոր դեպքերը։ Բանն այն է, որ
\ident{stdio} գրադարանում սահմանված են երեք ստանդարտ ֆայլային
դեսկրիպտորներ՝ \ident{stdin} -- ներմուծման, \ident{stdout} --
արտածման և \ident{stderr} -- սխալների հոսքերի համար։ Եվ
\ident{scanf} ֆունկցիան կարդում է \ident{stdin} հոսքից, իսկ
\ident{printf} ու \ident{puts} ֆունկցիաները գրում են
\ident{stdout} հոսքում։
